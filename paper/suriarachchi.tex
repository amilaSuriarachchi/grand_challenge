% THIS IS SIGPROC-SP.TEX - VERSION 3.1
% WORKS WITH V3.2SP OF ACM_PROC_ARTICLE-SP.CLS
% APRIL 2009
%
% It is an example file showing how to use the 'acm_proc_article-sp.cls' V3.2SP
% LaTeX2e document class file for Conference Proceedings submissions.
% ----------------------------------------------------------------------------------------------------------------
% This .tex file (and associated .cls V3.2SP) *DOES NOT* produce:
%       1) The Permission Statement
%       2) The Conference (location) Info information
%       3) The Copyright Line with ACM data
%       4) Page numbering
% ---------------------------------------------------------------------------------------------------------------
% It is an example which *does* use the .bib file (from which the .bbl file
% is produced).
% REMEMBER HOWEVER: After having produced the .bbl file,
% and prior to final submission,
% you need to 'insert'  your .bbl file into your source .tex file so as to provide
% ONE 'self-contained' source file.
%
% Questions regarding SIGS should be sent to
% Adrienne Griscti ---> griscti@acm.org
%
% Questions/suggestions regarding the guidelines, .tex and .cls files, etc. to
% Gerald Murray ---> murray@hq.acm.org
%
% For tracking purposes - this is V3.1SP - APRIL 2009

%\documentclass{acm_proc_article-sp}
\documentclass{sig-alternate}

\usepackage{algorithm}
\usepackage{algorithmic}

\begin{document}

\newfont{\mycrnotice}{ptmr8t at 7pt}
\newfont{\myconfname}{ptmri8t at 7pt}
\let\crnotice\mycrnotice%
\let\confname\myconfname%


\permission{Permission to make digital or hard copies of all or part of this work for personal or classroom use is granted without fee provided that copies are not made or distributed for profit or commercial advantage and that copies bear this notice and the full citation on the first page. Copyrights for components of this work owned by others than ACM must be honored. Abstracting with credit is permitted. To copy otherwise, or republish, to post on servers or to redistribute to lists, requires prior specific permission and/or a fee. Request permissions from Permissions@acm.org.}
\conferenceinfo{DEBS$^,$ 15,} {June 29 - July 3, 2015, OSLO, Norway.}
\copyrightetc{Copyright 2015 ACM \the\acmcopyr}
\crdata{978-1-4503-3286-6/15/06\ ...\$15.00.\\
http://dx.doi.org/10.1145/2611286.2611330.}

\title{DEBS Grand Challenge: A High-Throughput, Scalable Solution for Calculating Frequent Routes and Profitability of New York Taxis}
%
% You need the command \numberofauthors to handle the 'placement
% and alignment' of the authors beneath the title.
%
% For aesthetic reasons, we recommend 'three authors at a time'
% i.e. three 'name/affiliation blocks' be placed beneath the title.
%
% NOTE: You are NOT restricted in how many 'rows' of
% "name/affiliations" may appear. We just ask that you restrict
% the number of 'columns' to three.
%
% Because of the available 'opening page real-estate'
% we ask you to refrain from putting more than six authors
% (two rows with three columns) beneath the article title.
% More than six makes the first-page appear very cluttered indeed.
%
% Use the \alignauthor commands to handle the names
% and affiliations for an 'aesthetic maximum' of six authors.
% Add names, affiliations, addresses for
% the seventh etc. author(s) as the argument for the
% \additionalauthors command.
% These 'additional authors' will be output/set for you
% without further effort on your part as the last section in
% the body of your article BEFORE References or any Appendices.

\numberofauthors{2} %  in this sample file, there are a *total*
% of EIGHT authors. SIX appear on the 'first-page' (for formatting
% reasons) and the remaining two appear in the \additionalauthors section.
%
\author{
% You can go ahead and credit any number of authors here,
% e.g. one 'row of three' or two rows (consisting of one row of three
% and a second row of one, two or three).
%
% The command \alignauthor (no curly braces needed) should
% precede each author name, affiliation/snail-mail address and
% e-mail address. Additionally, tag each line of
% affiliation/address with \affaddr, and tag the
% e-mail address with \email.
%
% 1st. author
\alignauthor
Amila Suriarachchi\\
       \affaddr{Colorado State University}\\
       \affaddr{Fort Collins}\\
       \affaddr{Colorado}\\
       \email{amilas@cs.colostate.edu}
% 2nd. author
\alignauthor
Shrideep Pallickara\\
       \affaddr{Colorado State University}\\
       \affaddr{Fort Collins}\\
       \affaddr{Colorado}\\
       \email{shrideep@cs.colostate.edu}
}
% There's nothing stopping you putting the seventh, eighth, etc.
% author on the opening page (as the 'third row') but we ask,
% for aesthetic reasons that you place these 'additional authors'
% in the \additional authors block, viz.
% Just remember to make sure that the TOTAL number of authors
% is the number that will appear on the first page PLUS the
% number that will appear in the \additionalauthors section.

\maketitle
\begin{abstract}
Processing complex queries on unbounded event streams in real-time, is a challenge for many data processing systems. These systems are expected to process data with reduced latency to generate real-time events, and at high throughput to minimize the required hardware. In this regard, DEBS Grand Challenge 2015 focuses on evaluating two queries (frequent routes and  profitable cells) in real-time with low latency and high throughput. These queries involve processing windows of thousands of records. Firstly, such processing demands efficient data structures and algorithms to minimize the processing overhead. Secondly, the system should partition data to evaluate them in parallel to make it scalable.

In this paper, we present a set of data structures that we designed to evaluate the aforementioned queries with O(log n) time complexity and a data partitioning technique to evaluate them in parallel. We then evaluate our solution on a single machine as well as in a distributed setting in a commodity cluster of machines over a 1Gbps LAN. We were able to process the frequent routes query with the 173 million trips dataset within 5 minutes with less than 4 millisecond latency and the profitable cells query with same dataset within 11 minutes with less than 5 millisecond latency.
\end{abstract}

% A category with the (minimum) three required fields
\category{C.2.4}{Distributed Systems}{Distributed applications}
%A category including the fourth, optional field follows...
\keywords{Distributed Event Processing, Real-time analytics, Running median, Time Complexity, Event Stream Processing} % NOT required for Proceedings

\section{Introduction}

The Grand Challenge 2015 problem focuses on processing New York city taxi trip data for the year 2013. The area considered is a square of 150km x  150km. The  square is divided into smaller units called cells and a taxi trip is considered as moving from one cell (pickup cell) to another cell (dropoff cell). A route is uniquely identified by the pickup cell and the dropoff cell. The dataset consists of 173 million events and solutions are expected to replay these data and generate events that meet the criteria of two defined queries. 

The first query is to find the events that change the top ten routes within last 30 minutes and emit the new route set details along the original event. The given dataset contains 899010 routes and the 30 minute time window for this query accounts around 10000 events assuming an equal event distribution. This creates the possibility of processing several thousand route counts especially during peak hours. In other words we need to evaluate several thousand route counts with each event and find 10 top routes. A similar analysis can be made for the profitable cells query as well. In addition to that it requires to calculate the median fare for each cell. 

\subsection{Research Challenges}

If we consider an application that must process the 173 million trips dataset within an year, it would be a trivial task and several simple implementations would satisfy that requirement. However if we consider an application that must process these data within minutes with the added constraint of minimal latency to process each event, we observe following research challenges.
\begin{enumerate}
	\item \textit{Handling thousands of values in real time :}  An algorithm with O(n) time complexity will not scale for thousands. How can we design an algorithm with O(log n) time complexity?
	\item \textit{Parallel evaluation :}  This application does not inherently support parallel evaluation. Partitioning route sets and cell sets triggers events for subsets, but application requires the events with whole sets. Therefore is it possible to partition data and still generate required results to support parallelism?  
\end{enumerate}

\subsection{Methodology}

We started our work seeking answers for the above challenges. First we evaluated doubly linked list and heap based approaches to build a data structure with O(log n) time complexity. After several evaluations, we developed a new data structure called \textit{NodeList} that utilizes both these concepts. In order to find the running median with O(log n) time complexity, we experimented with an algorithm which uses priority queue and the heap data structure we developed. We decided to use former since it gave better performance with smaller window sizes. We then focused on parallelizing the application and designed a strategy to aggregate the subset results to find the final results. We also conducted several experiments to evaluate the scalability of our solution. 


\subsection{Contributions}
We observe following key contributions, which can be used in solving future DEBS challenges as well.
\begin{enumerate}
	\item We have designed a set of  generic data structures capable of inserting, updating, removing and retrieving top values (order can be defined using a comparator function) with O(log n) time complexity. 
	\item Using our data structures we have developed an algorithm to find the running median with O(log n) time complexity. One of the key challenges of DEBS 2014 \cite{jerzak2014debs}  was to find the median of a large time window. We believe our solution is useful in such situations.
	\item We have developed a scalable parallel evaluation technique to process queries efficiently. 
\end{enumerate}

\subsection{Paper Organization}
The remainder of this paper is organized as follows. Section 2 provides an in-depth description about the internal data structures and the algorithms used to process the events. Section 3 describes our scalable design to evaluate queries and associated problems. Section 4 proves the scalability of our solution with experimental results. Section 5 discusses the related work including some last year solutions. Section 6 summarizes key concepts and discusses their further applications. 

\section{Query Implementation}
In this section, we elaborate how we have implemented the query logic to achieve O(log(n)) time complexity. The 2015 Grand Challenge problem has two queries. First is to evaluate top ten most frequent routes within last 30 minutes and second is to evaluate top ten profitable areas within last 15 minutes. We can use a composite object to keep route details for each route and fare details to each cell to compare values among routes and cells. However, each time window can have thousands of route and cell details. Therefore an efficient data structure is required to add, remove, update and retrieve top ten values from a set of objects for a given time window efficiently. Further since each event is received with route and cell details, above mentioned operations should perform using route and cell details as the key. In order to support all these functions with O(log(n)) time complexity we developed a data structure called NodeList. Following parts of this section describes how we have achieve this time complexity with \textit{NodeList} data structure and how we have implemented the queries using that data structure.

\subsection{NodeList data structure}

\begin{table}
\centering
\caption{Operations of \textit{NodeList} Data Structure}
\begin{tabular}{|l|l|} \hline
Operation Signature & Time Complexity \\ \hline \hline
add(key : Object, value : NodeValue) & O(log(n)) \\ \hline 
get(key : Object) : NodeValue & O(1) \\ \hline
remove(key : Object) : NodeValue & O(log(n)) \\ \hline
decrementPosition(key : Object) & O(log(n)) \\ \hline
incrementPosition(key : Object) & O(log(n)) \\ \hline
getTopValues() : List<NodeValue> & O(1) \\ \hline
\end{tabular}
\label{nodelist_api}
\end{table}

Table \ref{nodelist_api} shows the operations of this data structure. The NodeValue is an interface to plug any user defined type as a value. For an example, the route query can use a RouteCount object to keep route count while profitable cells query can use CellProfict object to keep fare details. This interface contains a method called compare to define the order of the values. The NodeList data structure provides the notion of a sorted values starting from position 0 (highest value) to its users. The decrementPostion operation moves the corresponding value towards position 0 until the correct position according to sorted order. Therefore this method should be called after increasing a value of an existing object. Similarly incrementPosition operation can be used to move a value to correct position after decreasing the value of an existing object. GetTopValues operation returns the top ten values of the list. Query algorithm sections further elaborate usage of these methods.

One way of implementing this data structure is to use a heap. Heap supports adding and extracting maximum value operations in O(log(n)) time. However since this application requires top 10 values, we need to retrieve maximum value ten times and insert them back, if we just a heap. On the other hand if we use a doubly linked list to keep values, it will take O(n) time for  decrementPostion and   incrementPosition operations. Therefore we can use a doubly linked list to keep first 10 values (this makes time complexity for all operations with this data structure O(1)) and heap to keep other values to avoid above problems.  A Map which points to values can be used to retrieve values using a key.

NodeList data structure uses two data structures called OrderedList to keep first 10 values and DynamicHeap to keep other values. In this way it has to change only the last value of OrderedList and maximum value of Dynamic heap as required. Following sections describes how these structures implemented with their operations.

\subsection{OrderedList data structure}

\begin{table}
\centering
\caption{Operations of \textit{OrderedList} Data Structure}
\begin{tabular}{|l|l|} \hline
Operation Signature & Time Complexity \\ \hline \hline
add(key : Object, value : NodeValue) & O(n) \\ \hline
containsKey(key : Object) : boolean & O(1) \\ \hline
get(key : Object) : NodeValue & O(1) \\ \hline
incrementPosition(key : Object) & O(n) \\ \hline
decrementPosition(key : Object) & O(n) \\ \hline
remove(key : Object) : NodeValue & O(1) \\ \hline
getLastPosition() : int & O(1) \\ \hline
getLastKey() : Object & O(1) \\ \hline
\end{tabular}
\label{orderedlist_api}
\end{table}

Table \ref{orderedlist_api} shows the operations of the \textit{OrderedList} data structure. First six methods has the same meanings as the \textit{NodeList} data structure and last two can be used to get the size of the list and the value of the last object. We internally keep a counter and a pointer to tail to support these operations in O(1) time. Figure \ref{orderedlist_impl} shows the implementation details. The Map structure (an HashMap) is used to retrieve object pointers with O(1) expected time. Each node of the doubly linked list points to previous and next nodes while storing the \textit{NodeValue}.


\begin{figure}[!t]
        \centering
        \includegraphics[width=3.0in]{orderedlist.png}
        \caption{\textit{OrderedList} data structure implementation}
        \label{orderedlist_impl}
\end{figure}

\section{Query evaluation}

In this section, we explain how to evaluate queries with the given data set. For this query evaluation we can use a distributed event stream processing system with the process graph as shown in Figure \ref{twonode_graph} for both queries.

\begin{figure}[!t]
        \centering
        \includegraphics[width=3.0in]{twonode_graph.png}
        \caption{Two node process graph to evaluate queries}
        \label{twonode_graph}
\end{figure}

\textit{EventEmitter} reads the data file and generates events. For real implementation, each query has its own \textit{EventEmitter} which reads each line of the data file and generates an event with relevant information. \textit{QueryEvaluator} evaluates the query against the received event and generates top ten value changed event.  In general we can evaluate queries sequentially and parallel. In a sequential execution, all events goes to one \textit{QueryEvaluator} instance and in parallel execution, event stream is partitioned using the key field and evaluate simultaneously with different \textit{QueryEvaluator} instances.

\subsection{Sequential evaluation}

We can sequentially execute events either in a single Java Virtual Machine (JVM) or different JVMs (either in one machine or different machines) using TCP sockets to communicate between process instances. Figure \ref{sequential} shows such possible executions. 

\begin{figure}[!t]
        \centering
        \includegraphics[width=3.0in]{sequential.png}
        \caption{Different types of sequential execution}
        \label{sequential}
\end{figure}

First both processors can be executed in same JVM with same thread. In that case both processes executes sequentially. We can make two processes parallel by executing them with two threads and communicate using a queue in same JVM or communicate using a TCP connection for different JVMs. In all these cases \textit{QueryEvaluator} processes messages sequentially and becomes the bottleneck. Next we show how to parallel \textit{QueryEvaluator} process. 



\section{Experiments}
We conducted several experiments to study the scalability of our solution. Firstly we study the throughput and mean message latency variation with the number of \textit{QueryEvaluators} for both single JVM and distributed setups to find the scalability of the parallel evaluation. We measured the end to end latency. At the \textit{EventEmitter}, we set the time stamp for all events once the record is read. At the \textit{QueryEvaluator} we copy the original time stamp, if that event causes a top ten value change event. Total delay calculated at the \textit{ResultAggregator} just before writing the event. In an ideal system, we should observe linear throughput increment with a constant message latency. However there are some exceptions and we have provided our explanations on them. Finally we studied the throughput variation with the window size to study the suitability of our solution for higher window sizes. All these experiments were performed on a LAN with a network bandwidth of 1Gbps and all machines were Intel(R) Xeon(R) 2.4GHz, 4 core duo machines with 16 GB of memory. 

\subsection{Frequent routes query}

\begin{figure}[!t]
        \centering
        \includegraphics[width=3.0in]{routegraph.png}
        \caption{Process graph to evaluate top ten frequent routes.}
        \label{routegraph}
\end{figure}

Figure \ref{routegraph} shows the process graph used to process frequent route queries. \textit{RouteEventEmitter} reads the data file and emits events with a sequence number. \textit{RouteEmitter} contains a set of threads (equals to the number of \textit{RouteProcessors}) to emit events more efficiently. When processing each line first it reads the data line and creates a \textit{Route} object using longitude and latitude coordinates of pickup and dropoff locations. Then it distributes the events among the \textit{RouteProcessors} using the hash value of the \textit{Route} object. \textit{TopRouteProcessor} aggregates the results and saves them to a file.

We deployed this process graph to a distributed stream processing system we have developed, and  measured the throughput and the mean message latency varying the number of \textit{RouteProcessors} with the data set given (173 million records). In the local set up, we use a new thread to execute a new instance and in the distributed setup we add a new node to execute the new instance.

\begin{figure}[!t]
        \centering
        \includegraphics[width=3.0in]{throughput_route.png}
        \caption{Throughput variation with the number of \textit{RouteProcessor} instances.}
        \label{throughput_route}
\end{figure}

Figure \ref{throughput_route} shows the throughput variation. For single JVM setup throughput increases from one instance to two instances. This is due to higher parallelism system achieves with two threads. After that throughput slightly decreases due to higher context switching with many threads. Similarly for distributed setup initially throughput increases with the number of   \textit{RouteProcessors} and achieves a steady limit. We observe two reasons for this non scalability after initial stages. Firstly, \textit{RouteProcessor} is not a computing intensive process. Secondly, there is only one \textit{RouteEventEmitter} and the speed at which it can read and send messages becomes the bottleneck. Initial throughput for distributed setup is less, due to message serialization, message deserialization and TCP overheads. 

\begin{figure}[!t]
        \centering
        \includegraphics[width=3.0in]{latency_route.png}
        \caption{Mean latency variation with the number of \textit{RouteProcessor} instances.}
        \label{latency_route}
\end{figure}

Figure \ref{latency_route} shows the latency variation. For single JVM setup latency get reduced at 2 and slightly increases after that due to context switching overhead with higher number of threads. For distributed setup there is an high initial latency due to communication overheads. After that it comes to a constant value. 

\subsection{Profitable cells query}

\begin{figure}[!t]
        \centering
        \includegraphics[width=3.0in]{profitgraph.png}
        \caption{Process graph to evaluate top ten profitable cells.}
        \label{profictgraph}
\end{figure}

Figure \ref{profictgraph} shows the process graph for profitable cells query. \textit{ProfitEventEmitter} reads the data file and calculates the pickup cell and dropoff cell details using longitude and latitude coordinates. Then it creates two events one for dropoff part and other to pickup part, with different sequence numbers and distributes the events among \textit{ProfitCalculators} using the hash value of the cell. In this way cells are partitioned into different \textit{ProfitCalculator} instances and each instance gets all the events relevant to its' cell set. We use a set of threads (equals to the number of Profict Calculators) to emit data as in the previous query. Finally \textit{TopProfitProcessor} aggregates events and writes to a file. 

Similar to earlier experiment,  we deployed this process graph to our system and measured the throughput and the average message latency increasing the number of  \textit{ProfitCalculaor} instances for both single JVM setup and distributed setup to examine the scalability of our solution.

\begin{figure}[!t]
        \centering
        \includegraphics[width=3.0in]{throughput_profit.png}
        \caption{Throughput variation with the number of \textit{ProfitCalculator} instances.}
        \label{throughput_profit}
\end{figure}
 
Figure \ref{throughput_profit} shows the throughput variation for both setups. For single JVM set up initially throughput increases, due to increment of parallelism and comes to a steady state. For distributed setup  throughput increases linearly, although it has an initial lesser throughput compared to single JVM setup due to message serialization, message deserialization and TCP overheads. 

\begin{figure}[!t]
        \centering
        \includegraphics[width=3.0in]{latency_profit.png}
        \caption{Mean latency variation with the number of \textit{ProfitCalculator} instances.}
        \label{latency_profit}
\end{figure}

Figure \ref{latency_profit} shows the latency variation for both setups. For single JVM, latencies slightly increased with the addition of new instances as expected. But for distributed setups latency linearly increases with the addition of new nodes after 6 \textit{ProfitCalculator} instances. This latency increment is due to longer queues at the \textit{TopProfitProcessor}. In our message ordering process, we can send a message to \textit{TopProfitProcessor} only if all queues are not empty. This allows some processes with less top ten profitable events to delay other process messages. This problem can  possibly fixed by adding a limit to the message queue size. On the other hand in a real time system, this event ordering many not required since event processing happens at real time.

\subsection{Scalability with Window Size}

The grand challenge problem uses 15 minute window size for profitable cells query. However in a real world problems window sizes may required. Therefore finally we evaluated scalability of our solution with the window size.  For this experiment, we measured the throughput of the profitable cells query using a single JVM and three instances by varying the window size. Further we compared the throughput with our \textit{DynamicHeap} based solution which is not effective in lower window sizes as well.

\begin{figure}[!t]
        \centering
        \includegraphics[width=3.0in]{window.png}
        \caption{Throughput variation with the window size.}
        \label{window}
\end{figure}

Figure \ref{window} shows the results. For both solutions, throughput does not drop by half, even though the window size increases 64 times. And for large windows \textit{DynamicHeap} performs better than the \textit{PriorityQueue} based one. This is due to two reasons. Firstly, all operations of \textit{DynamicHeap} are O(log(n)). Secondly, out of 173 million records there are only around 18000 unique fares. Therefore \textit{DyamicHeap} based algorithm does not consume memory beyond that. However , since \textit{PriortyQueue} based implementation keeps all values in memory it consumes more memory for large windows.

\section{Related work}
DEBS Grand Challenge problem naturally fall into the distributed stream processing domain. Many such systems have been developed with different feature sets according to the applications they use. Yahoo S4 \cite{neumeyer2010s4} is used at Yahoo to calculate the click-through rate, Twitter Storm \cite{toshniwal2014storm} is used in Twitter to process tweets, MillWheel \cite{akidau2013millwheel} is used at Google to track trends in web queries. All these systems are based on the Actor model \cite{agha1985actors} and provide a generic API to implement logic to process data. Apart from basic communication these frameworks provide other features such as reliable message delivery, state persistence and dynamic membership management. Spark stream \cite{zaharia2012discretized} provides fault tolerance based on its Resilient Distributed Datasets \cite{zaharia2012resilient}.  

Several approaches has been developed for DEBS Grand Challenge 2014 \cite{jerzak2014debs}. Predictive Load Management in Smart Grid Environments \cite{mutschler2014predictive} provides a solution based on EventCore. That involved development of efficient algorithms to find median. Scalable Stateful Stream Processing for Samart Grids \cite{fernandez2014scalable} implemented their solution on SEEP \cite{castro2013integrating} a scalable fault tolerant framework. Solving the Grand Challenge Using an Opensource CEP Engine \cite{perera2014solving} solution is based on WSO2 Complex Event Processor. All these solutions partition the data to scale up the application, provide solutions to predict some missing data and provides algorithmic solutions for efficient query processing. Our solution shares the same ideas along these solutions while addressing the specific issues for 2015 Grand Challenge.


\section{Conclusion}

We propose a scalable solution for DEBS grand challenge problem 2015. Our queries execute with O(log(n)) time complexity and hence scales with window size. We propose a scalable evaluation technique to execute queries in a distributed setup. Since \textit{EventEmitter} partitions the events using the  hash code value of the key (\textit{Route} or \textit{Cell}), our solution can be used to process events with multiple \textit{EventEmitters} (multiple event streams) as well. Different levels of \textit{ResultAggregators} can be used to scales up aggregation process if required. We propose an algorithm to find the runtime median with O(log(n)) time complexity. This algorithm specially suites for finding median with repeating values. The time windows used in our solution has an O(1) time complexity but O(n) space complexity. This creates opportunity to research further on efficient memory management techniques for large window sizes.

\newpage
%
% The following two commands are all you need in the
% initial runs of your .tex file to
% produce the bibliography for the citations in your paper.
\bibliographystyle{abbrv}
\bibliography{paper}  % sigproc.bib is the name of the Bibliography in this case
% You must have a proper ".bib" file
%  and remember to run:
% latex bibtex latex latex
% to resolve all references
%
% ACM needs 'a single self-contained file'!
%
%APPENDICES are optional
%\balancecolumns
%Appendix A
% That's all folks!
\end{document}
