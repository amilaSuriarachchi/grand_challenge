\section{Related work}
DEBS Grand Challenge problem naturally fall into the distributed stream processing domain. Many such systems have been developed with different feature sets according to the applications they use. Yahoo S4 \cite{neumeyer2010s4} is used at Yahoo to calculate the click-through rate, Twitter Storm \cite{toshniwal2014storm} is used in Twitter to process tweets, MillWheel \cite{akidau2013millwheel} is used at Google to track trends in web queries. All these systems are based on the Actor model \cite{agha1985actors} and provide a generic API to implement logic to process data. Apart from basic communication these frameworks provide other features such as reliable message delivery, state persistence and dynamic membership management. Spark stream \cite{zaharia2012discretized} provides fault tolerance based on its Resilient Distributed Datasets \cite{zaharia2012resilient}.  

Several approaches has been developed for DEBS Grand Challenge 2014 \cite{jerzak2014debs}. Predictive Load Management in Smart Grid Environments \cite{mutschler2014predictive} provides a solution based on EventCore. That involved development of efficient algorithms to find median. Scalable Stateful Stream Processing for Samart Grids \cite{fernandez2014scalable} implemented their solution on SEEP \cite{castro2013integrating} a scalable fault tolerant framework. Solving the Grand Challenge Using an Opensource CEP Engine \cite{perera2014solving} solution is based on WSO2 Complex Event Processor. All these solutions partition the data to scale up the application, provide solutions to predict some missing data and provides algorithmic solutions for efficient query processing. Our solution shares the same ideas along these solutions while addressing the specific issues for 2015 Grand Challenge.

