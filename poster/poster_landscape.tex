\documentclass[landscape,paperwidth=42in, paperheight=36in, fontscale=0.292]{baposter}

\usepackage[plain,vlined]{algorithm2e}
\usepackage{times}
\usepackage{calc}
\usepackage{url}
\usepackage{graphicx}
\usepackage{amsmath}
\usepackage{amssymb}
\usepackage{relsize}
\usepackage{multirow}
\usepackage{booktabs}
\usepackage{wrapfig}
\usepackage{graphicx,caption}

\usepackage{graphicx}
\usepackage{multicol}
\usepackage[T1]{fontenc}
\usepackage{ae}

\graphicspath{{images/}}
\captionsetup[figure]{labelformat=empty}

 %%%%%%%%%%%%%%%%%%%%%%%%%%%%%%%%%%%%%%%%%%%%%%%%%%%%%%%%%%%%%%%%%%%%%%%%%%%%%%%%
 %%%% Some math symbols used in the text
 %%%%%%%%%%%%%%%%%%%%%%%%%%%%%%%%%%%%%%%%%%%%%%%%%%%%%%%%%%%%%%%%%%%%%%%%%%%%%%%%
 % Format 
 \newcommand{\RotUP}[1]{\begin{sideways}#1\end{sideways}}


 %%%%%%%%%%%%%%%%%%%%%%%%%%%%%%%%%%%%%%%%%%%%%%%%%%%%%%%%%%%%%%%%%%%%%%%%%%%%%%%%
 % Multicol Settings
 %%%%%%%%%%%%%%%%%%%%%%%%%%%%%%%%%%%%%%%%%%%%%%%%%%%%%%%%%%%%%%%%%%%%%%%%%%%%%%%%
 \setlength{\columnsep}{0.7em}
 \setlength{\columnseprule}{0mm}


 %%%%%%%%%%%%%%%%%%%%%%%%%%%%%%%%%%%%%%%%%%%%%%%%%%%%%%%%%%%%%%%%%%%%%%%%%%%%%%%%
 % Save space in lists. Use this after the opening of the list
 %%%%%%%%%%%%%%%%%%%%%%%%%%%%%%%%%%%%%%%%%%%%%%%%%%%%%%%%%%%%%%%%%%%%%%%%%%%%%%%%
 \newcommand{\compresslist}{%
 \setlength{\itemsep}{1pt}%
 \setlength{\parskip}{0pt}%
 \setlength{\parsep}{0pt}%
 }


 %%%%%%%%%%%%%%%%%%%%%%%%%%%%%%%%%%%%%%%%%%%%%%%%%%%%%%%%%%%%%%%%%%%%%%%%%%%%%%
 % Formating
 \newcommand{\Matrix}[1]{\begin{bmatrix} #1 \end{bmatrix}}
 \newcommand{\Vector}[1]{\begin{pmatrix} #1 \end{pmatrix}}

 \newcommand*{\norm}[1]{\mathopen\| #1 \mathclose\|}% use instead of $\|x\|$
 \newcommand*{\abs}[1]{\mathopen| #1 \mathclose|}% use instead of $\|x\|$
 \newcommand*{\normLR}[1]{\left\| #1 \right\|}% use instead of $\|x\|$

 \newcommand*{\SET}[1]  {\ensuremath{\mathcal{#1}}}
 \newcommand*{\FUN}[1]  {\ensuremath{\mathcal{#1}}}
 \newcommand*{\MAT}[1]  {\ensuremath{\boldsymbol{#1}}}
 \newcommand*{\VEC}[1]  {\ensuremath{\boldsymbol{#1}}}
 \newcommand*{\CONST}[1]{\ensuremath{\mathit{#1}}}

 \DeclareMathOperator*{\argmax}{arg\,max}
 \DeclareMathOperator*{\diag}{diag}
 \DeclareMathOperator*{\argmin}{arg\,min}
 \DeclareMathOperator*{\vectorize}{vec}
 \DeclareMathOperator*{\reshape}{reshape}

 %-----------------------------------------------------------------------------
 % Differentiation
 \newcommand*{\Nabla}[1]{\nabla_{\!#1}}

 \renewcommand*{\d}{\mathrm{d}}
 \newcommand*{\dd}{\partial}

 \newcommand*{\At}[2]{\ensuremath{\left.#1\right|_{#2}}}
 \newcommand*{\AtZero}[1]{\At{#1}{\pp=\VEC 0}}

 \newcommand*{\diffp}[2]{\ensuremath{\frac{\dd #1}{\dd #2}}}
 \newcommand*{\diffpp}[3]{\ensuremath{\frac{\dd^2 #1}{\dd #2 \dd #3}}}
 \newcommand*{\diffppp}[4]{\ensuremath{\frac{\dd^3 #1}{\dd #2 \dd #3 \dd #4}}}
 \newcommand*{\difff}[2]{\ensuremath{\frac{\d #1}{\d #2}}}
 \newcommand*{\diffff}[3]{\ensuremath{\frac{\d^2 #1}{\d #2 \d #3}}}
 \newcommand*{\difffp}[3]{\ensuremath{\frac{\dd\d #1}{\d #2 \dd #3}}}
 \newcommand*{\difffpp}[4]{\ensuremath{\frac{\dd^2\d #1}{\d #2 \dd #3 \dd #4}}}

 \newcommand*{\diffpAtZero}[2]{\ensuremath{\AtZero{\diffp{#1}{#2}}}}
 \newcommand*{\diffppAtZero}[3]{\ensuremath{\AtZero{\diffpp{#1}{#2}{#3}}}}
 \newcommand*{\difffAt}[3]{\ensuremath{\At{\difff{#1}{#2}}{#3}}}
 \newcommand*{\difffAtZero}[2]{\ensuremath{\AtZero{\difff{#1}{#2}}}}
 \newcommand*{\difffpAtZero}[3]{\ensuremath{\AtZero{\difffp{#1}{#2}{#3}}}}
 \newcommand*{\difffppAtZero}[4]{\ensuremath{\AtZero{\difffpp{#1}{#2}{#3}{#4}}}}

 %-----------------------------------------------------------------------------
 % Defined
 % How should the defined operator look like (:= or ^= ==)
 % (I want back my :=, it is so much better than ^= because (1) it has a
 % direction and (2) everyone here uses it.)
 %
 % Use :=
 %\newcommand*{\defined}{\ensuremath{\mathrel{\mathop{:}}=}}
 %\newcommand*{\definedRight}{\ensuremath{=\mathrel{\mathop{:}}}}
 % Use ^=
 \newcommand*{\defined}{\ensuremath{\triangleq}}
 \newcommand*{\definedRight}{\ensuremath{\triangleq}}
 % Use = with three bars
 %\newcommand*{\defined}{\ensuremath{?}}
 %\newcommand*{\definedRight}{\ensuremath{?}}

 %%%%%%%%%%%%%%%%%%%%%%%%%%%%%%%%%%%%%%%%%%%%%%%%%%%%%%%%%%%%%%%%%%%%%%%%%%%%%%
 % Symbols used in the paper

 %-----------------------------------------------------------------------------
 % The Methods
 \newcommand*{\ICIA}{\emph{ICIA}}
 \newcommand*{\CoDe}{\emph{CoDe}}
 \newcommand*{\LinCoDe}{\emph{LinCoDe}}
 \newcommand*{\CoNe}{\emph{CoNe}}
 \newcommand*{\CoLiNe}{\emph{CoLiNe}}
 \newcommand*{\LinCoLiNe}{\emph{LinCoLiNe}}

 % inter eye distance
 \newcommand*{\ied}{IED}

 %-----------------------------------------------------------------------------
 % Koerper
 %%\newcommand*{\RR}{\mathbb{R}}
 %\newcommand*{\RR}{{I\hspace{-3.5pt}R}}
 %\newcommand*{\RR}{{\mathrm{I\hspace{-2.7pt}R}}}

 \font\dsfnt=dsrom12

 \DeclareSymbolFont{nark}{U}{dsrom}{m}{n}
 \DeclareMathSymbol{\NN}{\dsfnt}{nark}{`N}
 \DeclareMathSymbol{\RR}{\dsfnt}{nark}{`R}
 \DeclareMathSymbol{\ZZ}{\dsfnt}{nark}{`Z}

 %-----------------------------------------------------------------------------
 % Domains
 \newcommand*{\D}{\mathcal{D}}
 \newcommand*{\I}{\mathcal{I}}

 %-----------------------------------------------------------------------------
 % Texture coordinates
 \newcommand*{\rr}{\VEC{r}}

 %-----------------------------------------------------------------------------
 % Parameters
 \newcommand*{\pt}{\VEC{\tau}}
 \newcommand*{\pr}{\VEC{\rho}}
 \newcommand*{\pp}{\VEC{p}}
 \newcommand*{\qq}{\VEC{q}}
 \newcommand*{\xx}{\VEC{x}}
 \newcommand*{\deltaq}{\Delta \qq}
 \newcommand*{\deltap}{\Delta \pp}
 \newcommand*{\zz}{\VEC{z}}
 \newcommand*{\pa}{\VEC{\alpha}}
 \newcommand*{\qa}{\VEC{\alpha}}
 \newcommand*{\pb}{\VEC{\beta}}

 %-----------------------------------------------------------------------------
 % Optimal appearance parameters
 \newcommand*{\pbh}[1]{\ensuremath{\hat{\pb}({#1})}}

 %-----------------------------------------------------------------------------
 % Warp basis
 \newcommand*{\M}[1]{\ensuremath{M({#1})}}
 \newcommand*{\LL}[1]{\ensuremath{L({#1})}}

 %-----------------------------------------------------------------------------
 % Matrices of the texture model
 \newcommand*{\AM}[1]{\ensuremath{\Lambda(#1)}}               % Lambda(beta) 
 \newcommand*{\AMr}[2]{\ensuremath{\Lambda(#1; #2)}}        % Lambda(r, beta)

 \newcommand*{\As}{A}         % Continuous Basis symbol
 \newcommand*{\afs}{a}        % Continuous mean symbol
 \newcommand*{\A}[1]{\As(#1)}         % Continuous Basis
 \newcommand*{\af}[1]{\afs(#1)}        % Continuous mean


 %-----------------------------------------------------------------------------
 % Matrices of the shape model
 \newcommand*{\MU}{\VEC{\mu}}
 \newcommand*{\MM}{\MAT{M}}

 %-----------------------------------------------------------------------------
 %% The project out matrix and operator
 \newcommand*{\INT}{\MAT{P}}
 \newcommand*{\INTf}{P}

 %-----------------------------------------------------------------------------
 % The identity matrix
 \newcommand*{\EYEtwo}{\Matrix{1 & 0\\0&1}}
 \newcommand*{\EYE}{\MAT E}
 \newcommand*{\EYEf}{E}

 % Wether to use subscripts or brackets for some function arguments
 % can be decided by commenting out the corresponding functions underneath
 %-----------------------------------------------------------------------------
 % Mapping
 \newcommand*{\Cs}[1]{\ensuremath{C^{#1}}} % C symbol
 \newcommand*{\C}[2]{\ensuremath{C^{#1}(#2)}} % Use C with brackets

 %-----------------------------------------------------------------------------
 % Objective function
 \newcommand*{\Fs}{\ensuremath{F}}              % F symbol
 \newcommand*{\F}[1]{\ensuremath{\Fs(#1)}}       % Use F with brackets    F(q)

 %-----------------------------------------------------------------------------
 % Approximated objective functions
 \newcommand*{\FFs}{\tilde{F}}                     % ~F symbol
 \newcommand*{\FF}[1]{\ensuremath{\FFs(#1)}}       % Use ~F with brackets    F(q)

 %-----------------------------------------------------------------------------
 % residual function
 \newcommand*{\es}{\ensuremath{f}}              % R symbol

 \newcommand*{\e}[1]{\ensuremath{\es(#1)}}         % R(q)
 \newcommand*{\er}[2]{\ensuremath{\es(#1; #2)}}    % R(r; q)

 %-----------------------------------------------------------------------------
 % Approximated residual functions
 \newcommand*{\ees}{\tilde{f}}                       % ~R symbol
 \newcommand*{\ee}[1]{\ensuremath{\ees(#1)}}       % ~R(q)
 \newcommand*{\eer}[2]{\ensuremath{\ees(#2; #1)}}  % ~R(r; q)

 %-----------------------------------------------------------------------------
 % Warps
 \newcommand*{\Vs}{\ensuremath{V}}
 \newcommand*{\VLins}{\ensuremath{\Vs^{\text{Ortho}}}}
 \newcommand{\VModels}{\ensuremath{\Vs^{\text{Model}}}}
 \newcommand*{\Ws}{\ensuremath{W}}

 \newcommand{\V}[1]{\ensuremath{\Vs(#1)}}
 \newcommand{\VModel}[1]{\ensuremath{\VModels(#1)}}
 \newcommand{\Vr}[2]{\ensuremath{\Vs(#1; #2)}}
 \newcommand{\VInvr}[2]{\ensuremath{\Vs^{-1}(#1; #2)}}
 \newcommand{\VrLin}[2]{\ensuremath{\VLins(#1; #2)}}
 \newcommand{\W}[1]{\ensuremath{\Ws(#1)}}
 \newcommand{\Winv}[1]{\ensuremath{\Ws^{-1}(#1)}}
 \newcommand{\Wr}[2]{\ensuremath{\Ws(#1; #2)}}


%%%%%%%%%%%%%%%%%%%%%%%%%%%%%%%%%%%%%%%%%%%%%%%%%%%%%%%%%%%%%%%%%%%%%%%%%%%%%
%% Begin of Document
%%%%%%%%%%%%%%%%%%%%%%%%%%%%%%%%%%%%%%%%%%%%%%%%%%%%%%%%%%%%%%%%%%%%%%%%%%%%%
\begin{document}
%%%%%%%%%%%%%%%%%%%%%%%%%%%%%%%%%%%%%%%%%%%%%%%%%%%%%%%%%%%%%%%%%%%%%%%%%%%%%
%% Here starts the poster
%%---------------------------------------------------------------------------
%% Format it to your taste with the options
%%%%%%%%%%%%%%%%%%%%%%%%%%%%%%%%%%%%%%%%%%%%%%%%%%%%%%%%%%%%%%%%%%%%%%%%%%%%%
\begin{poster}{
 % Show grid to help with alignment
 grid=false,
 % Column spacing
 colspacing=0.7em,
 % Color style
 headerColorOne=cyan!20!white!90!black,
 borderColor=cyan!30!white!90!black,
 % Format of textbox
 textborder=faded,
 % Format of text header
 headerborder=open,
 headershape=roundedright,
 headershade=plain,
 background=none,
 bgColorOne=cyan!10!white,
 headerheight=0.12\textheight}
 
 % Title
 {\Huge A High-throughput, Scalable Solution for the New York Taxi Trips DEBS Grand Challenge Problem}
 % Authors
 {Amila Suriarachchi\\[1em]
 {\texttt{amilas@cs.colostate.edu}}}
 % University logo
 {
  \begin{tabular}{r}
    \includegraphics[height=0.12\textheight]{logo.jpg}\\
  \end{tabular}
 }

%%%%%%%%%%%%%%%%%%%%%%%%%%%%%%%%%%%%%%%%%%%%%%%%%%%%%%%%%%%%%%%%%%%%%%%%%%%%%%
%%% Now define the boxes that make up the poster
%%%---------------------------------------------------------------------------
%%% Each box has a name and can be placed absolutely or relatively.
%%% The only inconvenience is that you can only specify a relative position 
%%% towards an already declared box. So if you have a box attached to the 
%%% bottom, one to the top and a third one which should be inbetween, you 
%%% have to specify the top and bottom boxes before you specify the middle 
%%% box.
%%%%%%%%%%%%%%%%%%%%%%%%%%%%%%%%%%%%%%%%%%%%%%%%%%%%%%%%%%%%%%%%%%%%%%%%%%%%%%

%%%%%%%%%%%%%%%%%%%%%%%%%%%%%%%%%%%%%%%%%%%%%%%%%%%%%%%%%%%%%%%%%%%%%%%%%%%%%%
  \headerbox{Abstract}{name=abstract,column=0,row=0,span=1}{
%%%%%%%%%%%%%%%%%%%%%%%%%%%%%%%%%%%%%%%%%%%%%%%%%%%%%%%%%%%%%%%%%%%%%%%%%%%%%%
DEBS Grand Challenge 2015 focuses on evaluating two queries (frequent routes and  profitable cells) in real time with low latency and high throughput. These queries involve processing windows of thousands of records. Firstly, such processing demands efficient data structures and algorithms to minimize the processing overhead. Secondly, the system should partition data to evaluate them in parallel to make it scalable.

In this paper, we present a set of data structures that we designed to evaluate the aforementioned queries with O(log n) time complexity and a data partitioning technique to evaluate them in parallel.
  
}
%%%%%%%%%%%%%%%%%%%%%%%%%%%%%%%%%%%%%%%%%%%%%%%%%%%%%%%%%%%%%%%%%%%%%%%%%%%%%%
  \headerbox{Problem}{name=problem,column=0,below=abstract}{
%%%%%%%%%%%%%%%%%%%%%%%%%%%%%%%%%%%%%%%%%%%%%%%%%%%%%%%%%%%%%%%%%%%%%%%%%%%%%%

The Grand Challenge 2015 problem focuses on processing New York city taxi trip data for the year 2013. The area considered is a square of 150km x  150km. The  square is divided into smaller units called cells and a taxi trip is considered as moving from one cell (pickup cell) to another cell (dropoff cell). A route is uniquely identified by the pickup cell and the dropoff cell. The dataset consists of 173 million events and solutions are expected to replay these data and find events that change the topten,
\begin{enumerate}
	\item routes within last 30 minutes and emit the new route set details along the original event.
	\item profitable cells within last 15 minutes and emit the new cell set details along the original event.
	
		$Profitability = \frac{Median fare}{Number of empty taxis}$
\end{enumerate}

}

%%%%%%%%%%%%%%%%%%%%%%%%%%%%%%%%%%%%%%%%%%%%%%%%%%%%%%%%%%%%%%%%%%%%%%%%%%%%%%
  \headerbox{Contributions}{name=contributions,column=0,below=problem}{
%%%%%%%%%%%%%%%%%%%%%%%%%%%%%%%%%%%%%%%%%%%%%%%%%%%%%%%%%%%%%%%%%%%%%%%%%%%%%%
\begin{enumerate}
        \item A set of  generic data structures capable of inserting, updating, removing and retrieving top values (order can be defined using a comparator function) with O(log n) time complexity.
        \item An algorithm to find the running median with O(log n) time complexity. 
        \item A scalable parallel evaluation technique to process queries efficiently.
\end{enumerate}
}

%%%%%%%%%%%%%%%%%%%%%%%%%%%%%%%%%%%%%%%%%%%%%%%%%%%%%%%%%%%%%%%%%%%%%%%%%%%%%%
  \headerbox{NodeList Data Structure}{name=nodelist,column=1,row=0, span=2}{
%%%%%%%%%%%%%%%%%%%%%%%%%%%%%%%%%%%%%%%%%%%%%%%%%%%%%%%%%%%%%%%%%%%%%%%%%%%%%%
\begin{minipage}[t]{0.6\linewidth}

\begin{center}
\begin{tabular}{l l}
\toprule
\textbf{Operation Signature} & \textbf{Time} \\
 & \textbf{Complexity} \\
\midrule
add(key : Object, value : NodeValue) & O(log n) \\ 
get(key : Object) : NodeValue & O(1) \\ 
remove(key : Object) : NodeValue & O(log n) \\ 
decrementPosition(key : Object) & O(log n) \\ 
incrementPosition(key : Object) & O(log n) \\ 
getTopValues() : List<NodeValue> & O(1) \\ 
\bottomrule
\end{tabular}
\captionof{table}{Operations of \textit{NodeList} Data Structure}
\end{center}

\end{minipage}
\hspace{1em plus 1fill}
  \begin{minipage}[t]{0.35\linewidth}
    \centering
   \vspace{-2ex}
   \includegraphics[width=0.8\linewidth, height=0.4\linewidth]{nodeliststruct.png}
   \\\smaller A Schematic Representation of \textit{NodeList} Data Structure
\end{minipage}

}

%%%%%%%%%%%%%%%%%%%%%%%%%%%%%%%%%%%%%%%%%%%%%%%%%%%%%%%%%%%%%%%%%%%%%%%%%%%%%%
  \headerbox{Internal Data Structures}{name=structures,column=1,row=0, span=2, below=nodelist}{
%%%%%%%%%%%%%%%%%%%%%%%%%%%%%%%%%%%%%%%%%%%%%%%%%%%%%%%%%%%%%%%%%%%%%%%%%%%%%%
\begin{minipage}[t]{0.45\linewidth}
    \centering
    \vspace{1ex}
    \includegraphics[width=0.8\linewidth, height=0.5\linewidth]{orderedlist.png}
    \\\smaller A Schematic Representation of \textit{OrderList} Data Structure
\end{minipage}
\begin{minipage}[t]{0.45\linewidth}
    \centering
    \vspace{1ex}
    \includegraphics[width=0.8\linewidth, height=0.5\linewidth]{DinamicHeap.png}
    \\\smaller A Schematic Representation of \textit{DynamicHeap} Data Structure
\end{minipage}
}

%%%%%%%%%%%%%%%%%%%%%%%%%%%%%%%%%%%%%%%%%%%%%%%%%%%%%%%%%%%%%%%%%%%%%%%%%%%%%%
  \headerbox{Query Evaluation Techniques}{name=techniques,column=1,row=0, span=2, below=structures}{
%%%%%%%%%%%%%%%%%%%%%%%%%%%%%%%%%%%%%%%%%%%%%%%%%%%%%%%%%%%%%%%%%%%%%%%%%%%%%%

\begin{minipage}[t]{0.45\linewidth}
    \centering
    \vspace{1ex}
    \includegraphics[width=0.8\linewidth, height=0.5\linewidth]{sequential.png}
    \\\smaller Different types of sequential execution
\end{minipage}
\begin{minipage}[t]{0.45\linewidth}
    \centering
    \vspace{1ex}
    \includegraphics[width=0.8\linewidth, height=0.5\linewidth]{parallel.png}
    \\\smaller Different types of parallel execution
\end{minipage}


}
%%%%%%%%%%%%%%%%%%%%%%%%%%%%%%%%%%%%%%%%%%%%%%%%%%%%%%%%%%%%%%%%%%%%%%%%%%%%%%
  \headerbox{Scalability Results}{name=results,column=1,row=0, span=2, below=techniques}{
%%%%%%%%%%%%%%%%%%%%%%%%%%%%%%%%%%%%%%%%%%%%%%%%%%%%%%%%%%%%%%%%%%%%%%%%%%%%%%
Following graphs depict the throughput and mean message latency variation with the number of \textit{QueryEvaluators} for both single JVM and distributed setups to access the scalability of the parallel evaluation. In an ideal system, a linear throughput increment with a constant message latency should observed.

\begin{tabular}{@{}c@{ }c@{ }c@{ }c@{}@{ }@{ }c@{ }c@{ }c@{ }c@{ }}
    \includegraphics[width=0.24\linewidth]{time_route.png}&
    \includegraphics[width=0.24\linewidth]{latency_route.png}&
    \includegraphics[width=0.24\linewidth]{time_profit.png} &
    \includegraphics[width=0.24\linewidth]{latency_profit.png} \\[0.2em]
    \smaller Throughput variation  & \smaller Mean latency variation  & \smaller Throughput variation & \smaller Mean latency variation \\ 
    \smaller with the number of & \smaller  with the number of & \smaller  with the number of & \smaller  with the number of \\
    \smaller \textit{RouteProcessor} instances & \smaller \textit{RouteProcessor} instances & \smaller \textit{ProfitCalculator} instances & \smaller \textit{ProfitCalculator} instances \\
\end{tabular}

}

%%%%%%%%%%%%%%%%%%%%%%%%%%%%%%%%%%%%%%%%%%%%%%%%%%%%%%%%%%%%%%%%%%%%%%%%%%%%%%
  \headerbox{Frequent Routes Query}{name=frq,column=3, row=0}{
%%%%%%%%%%%%%%%%%%%%%%%%%%%%%%%%%%%%%%%%%%%%%%%%%%%%%%%%%%%%%%%%%%%%%%%%%%%%%%
\begin{algorithm}[H]
before $\gets$ nodeList.getTopValues()\;
window.add(event)\;
\eIf{ nodeList.containsKey(event.route) }{
        routCount $\gets$ nodeList.get(event.route)\;
        routCount.count++\;
        nodeList.decrementPosition (event.route)\;
}{
        nodeList.add(event.route, new RouteCount(1))\;
}

\While{ there exists expired events }{
        expEvent $\gets$ window.poll()\;
        routCount $\gets$ nodeList.get(expEvent.route)\;
        routCount.count--\;
        \eIf{ routCount.count is 0 }{
                nodelist.remove(expEvent.route)\;
        }{
                nodeList.incrementPosition (expEvent.route)\;
        }
}

now $\gets$ nodeList.getTopValues()\;
  
\If{ before not equals now }{
        create new top ten route event\;
}

\end{algorithm}
}   

%%%%%%%%%%%%%%%%%%%%%%%%%%%%%%%%%%%%%%%%%%%%%%%%%%%%%%%%%%%%%%%%%%%%%%%%%%%%%%
  \headerbox{ResultAggregator}{name=aggregator,column=3,row=0, below=frq}{
%%%%%%%%%%%%%%%%%%%%%%%%%%%%%%%%%%%%%%%%%%%%%%%%%%%%%%%%%%%%%%%%%%%%%%%%%%%%%%
\begin{algorithm}[H]
before $\gets$ nodeList.getTopValues()\;
\For {key in event.removedKeys}{
        nodeList.remove(key)\;
}
\For {value in event.newList}{
        \eIf {nodeList.containsKey(value.key)}{
                current $\gets$ nodeList.get(value.key)\;
                \eIf { current is lesser than value }{
                        update the current value \;
                        nodeList.decrementPosition (value.key) \;
                }{
                        update the current value \;
                        nodeList.incrementPosition (value.key) \;
                }
        }{
                nodeList.add(value.key, value)\;
        }
}
now $\gets$ nodeList.getTopValues()\;
\If { before not equals now }{
        emit new event
}
\end{algorithm}

}
\end{poster}%
%
\end{document}
